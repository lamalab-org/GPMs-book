
\documentclass[unijena]{scrlttr2}
\usepackage[utf8]{inputenc}

\setkomavar{fromfaculty}{Helmholtz Institute for Polymers in Energy Applications (HIPOLE Jena) \\
% Center for Energy and Environmental Chemistry (CEEC) \\ 
% Laboratory of Organic and Macromolecular
% Chemistry (IOMC)
}
\setkomavar{fromfacultyshort}{HIPOLE Jena}
\setkomavar{fromname}{Dr. Kevin Maik Jablonka}
\setkomavar{fromposition}{Research group leader}
\setkomavar{fromaddress}{Lessingstrasse 12--14 \\ 
07743 Jena}
% \setkomavar{fromphone}{+49 3641 9-48238}
\setkomavar{fromemail}{kevin.jablonka@uni-jena.de}
\setkomavar{fromurl}{lamalab.org}
\usepackage{csquotes}
\usepackage{hyperref}
\begin{document}
\begin{letter}{%
\hfill
}
\selectlanguage{english}
\setkomavar{subject}{Submission of our article \enquote{General purpose models for the chemical sciences} for consideration in \textit{Chemical Reviews}}

\opening{Dear editor,}

foundation models, such as large language models (LLMs), have been transforming many parts of society. 
This does not exclude the chemical sciences. In almost all steps of the scientific process applications of such models have been reported. Some of those reports even describe end-to-end automation of the entire scientific process. 

Thus, we believe it is timely to provide a technical background and overview on such models. Our review article \enquote{General purpose models for the chemical sciences} attempts to do that. 
We delineate different kinds of foundation models, provide first-principles explanations of their building principles and critically describe their applications in the chemical sciences. 
Given the transformative potential of these models, we also discuss safety and ethical implications in depth. 

Our review is, to our knowledge, the first of its kind and all figures have been created from scratch for this article. We expect it to be of much interest for the readership of \textit{Chemical Reviews}.

\closing{Sincerely,\vspace{3em}}




\end{letter}
\end{document}
 